\chapter*{Введение}
\addcontentsline{toc}{chapter}{Введение}

Термин «матрица» применяется во множестве разных областей: от
программирования до кинематографии.

Матрица в математике -- это таблица чисел, состоящая из определенного количества строк (m) и столбцов (n).

Мы встречаемся с матрицами каждый день, так как любую числовую информацию, структурированную в виде таблицы, можно рассматривать как матрицу.


Целью работы является изучение и реализация алгоритмов
умножения матриц, вычисление трудоёмкости этих алгоритмов. В данной
лабораторной работе рассматривается стандартный алгоритм умножения
матриц, алгоритм Винограда и модифицированный алгоритм Винограда.


Задачи данной лабораторной:

\begin{enumerate}[label=\arabic*)]
	\item Реализовать алгоритмы перемножения матриц:
	\begin{enumerate}
		\item[1.1)] классический;
		\item[1.2)] Винограда;
		\item[1.3)] Винограда с оптимизациями согласно варианту;
	\end{enumerate}
	\item оценить трудоемкость алгоритмов;
	\item провести замеры времени работы и объема используемой памяти алгоритмов;
	\item предложить рекомендации об особенностях применения оптимизаций.
\end{enumerate}
