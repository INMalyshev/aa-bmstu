\chapter{Технологическая часть}

В данном разделе будут приведены требования к программному обеспечению, средства реализации и листинги кода.

\section{Требования к программе}

К программе предъявляется ряд требований.
\begin{itemize}
	\item Входными данными являются две матрицы A и B. 
	\item Количество столбцов матрицы A долджно быть равно количеству строк матрицы B.
	\item На выходе получается результат умножения, введенных пользователем, матриц.

\end{itemize}


\section{Средства реализации}

В качестве языка программирования для реализации данной лабораторной работы был выбран язык программирования (ЯП) Python \cite{pythonlang}. 

Данный ЯП предоставляет широкий набор библиотек и руководств по их использованию, что значительно облегчает выполнения большинства задач в области программирования. 

Время работы алгоритмов было замерено с помощью функции time() из библиотеки time \cite{pythonlangtime}.



\section{Сведения о модулях программы}
Программа состоит из двух модулей.
\begin{enumerate}
	\item main.py - главный файл программы, в котором располагаются коды всех алгоритмов и меню.
	\item test.py - файл с замерами времени для графического изображения результата.
\end{enumerate}


\section{Листинг кода}

В листинге \ref{lst:standart} приведена реализация алгоритма стандартного алгоритма матриц.

В листингах \ref{lst:vinograd}, \ref{lst:vinograd_opt} приведена реализация алгоритма Копперсмита --- Винограда.


\begin{lstlisting}[label=lst:standart,caption=Реализация стандарного умножения матриц]
def simple_matrix_mult(matrix1, matrix2):
	if (len(matrix1) != len(matrix2[0])):
		print("Error size matrix!")
		return -1
	
	n = len(matrix1)
	m = len(matrix1[0])
	q = len(matrix2[0])
	
	matrix_res = [[0] * q for i in range(n)]
	
	for i in range(n):
		for j in range(q):
			for k in range(m):
				matrix_res[i][j] = matrix_res[i][j] + \
					matrix1[i][k] * matrix2[k][j]
				
	return matrix_res
	
\end{lstlisting}

\begin{lstlisting}[label=lst:vinograd,caption=Реализация алгоритма Копперсмита-Винограда]
def winograd_matrix_mult(matrix1, matrix2):
	if (len(matrix2) != len(matrix1[0])):
		print("Error size matrix!")
		return -1
	
	n = len(matrix1)
	m = len(matrix1[0])
	q = len(matrix2[0])
	
	matrix_res = [[0] * q for i in range(n)]
	
	row_factor = [0] * n
	for i in range(n):
		for j in range(0, m // 2, 1):
			row_factor[i] = row_factor[i] + \
				matrix1[i][2 * j] * matrix1[i][2 * j + 1]
	
	column_factor = [0] * q
	for i in range(q):
		for j in range(0, m // 2, 1):
			column_factor[i] = column_factor[i] + \
				matrix2[2 * j][i] * matrix2[2 * j + 1][i]
	
	for i in range(n):
		for j in range(q):
			matrix_res[i][j] = -row_factor[i] - column_factor[j]
			for k in range(0, m // 2, 1):
				matrix_res[i][j] = matrix_res[i][j] +\
					(matrix1[i][2 * k + 1] + matrix2[2 * k][j]) * \
					(matrix1[i][2 * k] + matrix2[2 * k + 1][j])
	
	if m % 2 == 1:
		for i in range(n):
			for j in range(q):
				matrix_res[i][j] = matrix_res[i][j] +\
					matrix1[i][m - 1] * matrix2[m - 1][j]
	
	return matrix_res
\end{lstlisting}

\begin{lstlisting}[label=lst:vinograd_opt,caption=Реализация алгоритма Копперсмита-Винограда (оптимизированный)]
def winograd_matrix_mult_opim(matrix1, matrix2):
	if (len(matrix2) != len(matrix1[0])):
		print("Error size matrix!")
		return -1
	
	n = len(matrix1)
	m = len(matrix1[0])
	q = len(matrix2[0])
	
	matrix_res = [[0] * q for i in range(n)]
	
	row_factor = [0] * n
	for i in range(n):
		for j in range(1, m, 2):
			row_factor[i] += matrix1[i][j] * matrix1[i][j - 1]
	
	column_factor = [0] * q
	for i in range(q):
		for j in range(1, m, 2):
			column_factor[i] += matrix2[j][i] * matrix2[j - 1][i]
	
	flag = n % 2
	for i in range(n):
		for j in range(q):
			matrix_res[i][j] = -(row_factor[i] + column_factor[j])
			for k in range(1, m, 2):
				matrix_res[i][j] += (matrix1[i][k - 1] + matrix2[k][j]) * \
						(matrix1[i][k] + matrix2[k - 1][j])
			if (flag):
				matrix_res[i][j] += matrix1[i][m - 1] \
					* matrix2[m - 1][j]

	return matrix_res
\end{lstlisting}


\section{Функциональные тесты}

В таблице~\ref{tabular:test_rec} приведены тесты для функций, реализующих стандартный алгоритм умножения матриц, алгоритм Винограда и оптимизированный алгоритм Винограда. Тесты пройдены успешно.

\clearpage

\begin{table}[h!]
	\captionsetup{justification=raggedright,singlelinecheck=off}
	\caption{\label{tabular:test_rec} Тестирование функций}
	\begin{center}
		\begin{tabular}{c@{\hspace{7mm}}c@{\hspace{7mm}}c@{\hspace{7mm}}c@{\hspace{7mm}}c@{\hspace{7mm}}c@{\hspace{7mm}}}
			\hline
			Матрица 1 & Матрица 2 &Ожидаемый результат \\ 
			\hline
			\vspace{4mm}
			$\begin{pmatrix}
				1 & 1 & 1\\
				5 & 5 & 5\\
				2 & 2 & 2
			\end{pmatrix}$ &
			$\begin{pmatrix}
				1 \\
				1 \\
				1 
			\end{pmatrix}$ &
			$\begin{pmatrix}
				3 \\
				15 \\
				6
			\end{pmatrix}$ \\
			\vspace{2mm}
			\vspace{2mm}
			$\begin{pmatrix}
				1 & 1 & 1
			\end{pmatrix}$ &
			$\begin{pmatrix}
				1 \\
				1 \\
				1
			\end{pmatrix}$ &
			$\begin{pmatrix}
				1 & 1 & 1\\
				1 & 1 & 1 \\
				1 & 1 & 1
			\end{pmatrix}$ \\
			\vspace{2mm}
			\vspace{2mm}
			$\begin{pmatrix}
				1 & 1 \\
				1 & 1
			\end{pmatrix}$ &
			$\begin{pmatrix}
				1 & 1\\
				1 & 1
			\end{pmatrix}$ &
			$\begin{pmatrix}
				2 & 2\\
				2 & 2
			\end{pmatrix}$ \\
			\vspace{2mm}
			\vspace{2mm}
			$\begin{pmatrix}
				2
			\end{pmatrix}$ &
			$\begin{pmatrix}
				2
			\end{pmatrix}$ &
			$\begin{pmatrix}
				4
			\end{pmatrix}$ \\
			\vspace{2mm}
			\vspace{2mm}
			$\begin{pmatrix}
				1 & -2 & 3\\
				1 & 2 & 3\\
				1 & 2 & 3
			\end{pmatrix}$ &
			$\begin{pmatrix}
				-1 & 2 & 3\\
				1 & 2 & 3\\
				1 & 2 & 3
			\end{pmatrix}$ &
			$\begin{pmatrix}
				0 & 4 & 6\\
				4 & 12 & 18\\
				4 & 12 & 18
			\end{pmatrix}$\\
			\vspace{2mm}
			\vspace{2mm}
			$\begin{pmatrix}
				1 & 2
			\end{pmatrix}$ &
			$\begin{pmatrix}
				1 & 2
			\end{pmatrix}$ &
			Неверный размер\\
		\end{tabular}
	\end{center}
	
\end{table}


\section*{Вывод}

В этом разделе была представлена реализация алгоритмов классического умножения матриц, алгоритма Винограда, оптимизирвоанного алгоритма Винограда. Тестирование программы было пройдено успешно.
