\chapter{Технологическая часть}

В данном разделе будут приведены требования к программному обеспечению, средства реализации и листинги кода.

\section{Требования к ПО}

Программа принимает две строки (регистрозависимые). В качестве результата возвращается число, равное редакторскому расстоянию. Необходимо реализовать возможность подсчета процессорнго времени и пиковой использованной памяти для каждого из алго­ритмов.

\section{Средства реализации}

Для реализации данной лабораторной работы был выбран язык программирования (ЯП) Python \cite{pythonlang}. 

Данный язык достаточно удобен и гибок в использовании. 

Время работы алгоритмов было замерено с помощью функции time() из библиотеки time \cite{pythonlangtime}

\section{Сведения о модулях программы}
Программа состоит из пяти модулей:
\begin{enumerate}
	\item algorithm.py - хранит реализацию алгоритмов;
	\item test.py - хранит реализацию тестирующей системы и тесты;
	\item memory.py - хранит реализацию системы замера памяти;
	\item time.py - хранит реализацию системы замера времени;
	\item utils.py - хранит реализацию вспомогательных функций.
\end{enumerate}


\section{Реализация алгоритмов}

 В листингах \ref{lst:non_rec_l}, \ref{lst:non_rec_dl}, \ref{lst:rec_dl}, \ref{lst:rec_dl_cache} приведены реализации алгоритмов нахождения расстояния Левенштейна и Дамерау -- Левенштейна.

\begin{lstlisting}[label=lst:non_rec_l,caption=Функция нахождения расстояния Левенштейна нерекурсивным методом.]
def non_recursive_levenshtein(s1, s2):
    distances = [[0 for _ in range(len(s1) + 1)] for __ in range(len(s2) + 1)]

    for i in range(len(s1) + 1):
        for j in range(len(s2) + 1):
            cur_dist = -1
            if i == 0 and j == 0:
                cur_dist = 0
            elif i == 0 and j > 0:
                cur_dist = j
            elif j == 0 and i > 0:
                cur_dist = i
            else:
                left = distances[j][i - 1] + 1
                up = distances[j - 1][i] + 1
                left_up = distances[j - 1][i - 1] + (0 if s1[i - 1] == s2[j - 1] else 1)
                cur_dist = min(left, up, left_up)
            
            distances[j][i] = cur_dist

    return distances[len(s2)][len(s1)]
	
\end{lstlisting}

\begin{lstlisting}[label=lst:non_rec_dl,caption=Функция нахождения расстояния Дамерау -- Левенштейна нерекурсивным методом.]
def non_recursive_damerau_levenshtein(s1, s2):
    distances = [[0 for _ in range(len(s1) + 1)] for __ in range(len(s2) + 1)]

    for i in range(len(s1) + 1):
        for j in range(len(s2) + 1):
            cur_dist = -1
            if i == 0 and j == 0:
                cur_dist = 0
            elif i == 0 and j > 0:
                cur_dist = j
            elif j == 0 and i > 0:
                cur_dist = i
            else:
                left = distances[j][i - 1] + 1
                up = distances[j - 1][i] + 1
                left_up = distances[j - 1][i - 1] + (0 if s1[i - 1] == s2[j - 1] else 1)
                if i > 1 and j > 1 and s1[i - 1] == s2[j - 2] and s1[i - 2] == s2[j - 1]:
                    exchange = distances[j - 2][i - 2] + 1
                    cur_dist = min(left, up, left_up, exchange)
                else:
                    cur_dist = min(left, up, left_up)
                
            distances[j][i] = cur_dist

    return distances[len(s2)][len(s1)]
	
\end{lstlisting}

\begin{lstlisting}[label=lst:rec_dl,caption=Функция нахождения расстояния Дамерау -- Левенштейна с использованием рекурсии.]
def recursive_damerau_levenshtein(s1, s2, i, j):
    if i == 0 and j == 0:
        return 0
    elif i == 0 and j > 0:
        return j
    elif j == 0 and i > 0:
        return i
    else:
        left = recursive_damerau_levenshtein(s1, s2, i - 1, j) + 1
        up = recursive_damerau_levenshtein(s1, s2, i, j - 1) + 1
        left_up = recursive_damerau_levenshtein(s1, s2, i - 1, j - 1) + (0 if s1[i - 1] == s2[j - 1] else 1)
        if i > 1 and j > 1 and s1[i - 1] == s2[j - 2] and s1[i - 2] == s2[j - 1]:
            exchange = recursive_damerau_levenshtein(s1, s2, i - 2, j - 2) + 1 if (i > 0 and j > 0 \
                and s1[i - 1] == s2[j - 2] and s1[i - 2] == s2[j - 1]) else left_up
            return min(left, up, left_up, exchange)
        else:
            return min(left, up, left_up)
	
\end{lstlisting}

\begin{lstlisting}[label=lst:rec_dl_cache,caption=Функция нахождения расстояния Дамерау -- Левенштейна рекурсивным методом с использованием кеша.]
def recursive_damerau_levenshtein_with_cache(s1, s2, i, j, cache):
    if cache[j][i] != -1:
        return cache[j][i]
    
    if i == 0 and j == 0:
        cache[j][i] = 0
        return cache[j][i]
    elif i == 0 and j > 0:
        cache[j][i] = j
        return cache[j][i]
    elif j == 0 and i > 0:
        cache[j][i] = i
        return cache[j][i]
    else:
        left = recursive_damerau_levenshtein_with_cache(s1, s2, i - 1, j, cache) + 1
        up = recursive_damerau_levenshtein_with_cache(s1, s2, i, j - 1, cache) + 1
        left_up = recursive_damerau_levenshtein_with_cache(s1, s2, i - 1, j - 1, cache)\
            + (0 if s1[i - 1] == s2[j - 1] else 1)
        if i > 1 and j > 1 and s1[i - 1] == s2[j - 2] and s1[i - 2] == s2[j - 1]:
            exchange = recursive_damerau_levenshtein_with_cache(s1, s2, i - 2, j - 2, cache)\
                + 1 if (i > 0 and j > 0 and s1[i - 1] == s2[j - 2] and s1[i - 2] == s2[j - 1]) else left_up
            cache[j][i] = min(left, up, left_up, exchange)
        else:
            cache[j][i] = min(left, up, left_up)
        
        return cache[j][i]
\end{lstlisting}

\section{Функциональные тесты}
В таблице \ref{tabular:functional_test} приведены функциональные тесты для алгоритмов вычисления расстояния Левенштейна (в таблице столбец подписан "Левенштейн") и Дамерау -- Левенштейна (в таблице - "Дамерау-Л."). Все тесты пройдены успешно.


\begin{table}[h]
	\begin{center}
		\captionsetup{justification=raggedright,singlelinecheck=off}
		\caption{\label{tabular:functional_test} Функциональные тесты}
		\begin{tabular}{|c|c|c|c|c|}
			\hline
			& & & \multicolumn{2}{c|}{Ожидаемый результат} \\
			\hline
			№&Строка 1&Строка 2&Левенштейн&Дамерау-Л. \\
			\hline
			1&"пустая строка"&"пустая строка"&0&0 \\
			\hline
			2&aaa&aaa&0&0 \\
			\hline
			3&"пустая строка"&asd&3&3 \\
			\hline
			4&asd&пустая строка"&3&3 \\
			\hline
			5&aaa&aaaaa&2&2 \\
			\hline
			8&as&sa&2&1 \\
			\hline
			9&asas&sasa&4&2 \\
			\hline
			10&asas&saas&2&1 \\
			\hline
			11&asas&saasa&3&2 \\
			\hline
		\end{tabular}
	\end{center}
\end{table}


\section*{Вывод}

Были разработаны и протестированы алгоритмы: нахождения расстояния Левенштейна нерекурсивно, нахождения расстояния Дамерау -- Левенштейна нерекурсивно, рекурсивно, а также рекурсивно с кешированием.
