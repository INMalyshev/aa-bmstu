\chapter*{Введение}
\addcontentsline{toc}{chapter}{Введение}

Целью данной лабораторной работы является изучение и исследование особенностей задач динамического программирования.

В качестве исследуемых задач, взяты нахождение расстояний Левенштейна и Дамерау -- Левенштейна.

\textbf{Расстояние Левенштейна}  (редакционное расстояние, дистанция редактирования) — метрика, измеряющая разность между двумя последовательностями символов. Она определяется как минимальное количество односимвольных операций (вставки, удаления, замены), необходимых для превращения одной строки в другую. В общем случае, операциям, используемым в этом преобразовании, можно назначить разные цены. Широко используется в теории информации и компьютерной лингвистике.

Расстояние Левенштейна \textbf{Расстояние Левенштейна \cite{Levenshtein}} и его обобщения активно применяются для:
\begin{itemize}
	\item исправления ошибок в слове (в поисковых системах, базах данных, при вводе текста, при автоматическом распознавании отсканированного текста или речи);
	\item сравнения текстовых файлов утилитой \code{diff} и ей подобными (здесь роль «символов» играют строки, а роль «строк» — файлы);
	\item сравнения генов в биоинформатике.
\end{itemize}

\textbf{Расстояние Дамерау -- Левенштейна} (названо в честь учёных Фредерика Дамерау и Владимира Левенштейна) — это мера разницы двух строк символов, определяемая как минимальное количество операций вставки, удаления, замены и транспозиции (перестановки двух соседних символов), необходимых для перевода одной строки в другую. Является модификацией расстояния Левенштейна, так как к операциям вставки, удаления и замены символов, определённых в расстоянии Левенштейна добавлена операция транспозиции (перестановки) символов.

\newpage

Задачами данной лабораторной являются:

\begin{enumerate}[label=\arabic*)]
	\item выбрать инструменты для замера процессорного времени выполнения реализаций алгоритмов;
	\item создать ПО, реализующее нерекурсивный метод поиска расстояния Левенштейна, нерекурсивный метод поиска Дамерау -- Левенштейна, рекурсивный метод поиска Дамерау -- Левенштейна, рекурсивный с кешированием метод поиска Дамерау -- Левенштейна;
	\item провести анализ затрат работы программы по времени и по памяти, выяснить влияющие на них характеристики;
	\item создать отчёт, содержащий:
	\begin{enumerate}
		\item схемы выбранных алгоритмов;
		\item обоснование выбора технических средств;
		\item результаты замеров времени и памяти реализации;
		\item обобщающий вывод.
	\end{enumerate}
\end{enumerate}
