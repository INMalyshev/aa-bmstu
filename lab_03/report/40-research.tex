\chapter{Исследовательская часть}

В данном разделе будут приведены примеры работы программ, постановка эксперимента и сравнительный анализ алгоритмов на основе полученных данных.

\section{Технические характеристики}

Технические характеристики устройства, на котором выполнялось тестирование, следующие:

\begin{itemize}
	\item Операционная система Ubuntu 22.04.1 \cite{ubuntu} Linux x86\_64.
	\item Память: 8 ГБ.
	\item Процессор: AMD® Ryzen 5 3500u.
\end{itemize}

Тестирование проводилось на ноутбуке, включенном в сеть электропитания. Во время тестирования ноутбук был нагружен только встроенными приложениями окружения, а также непосредственно системой тестирования.

\section{Время выполнения алгоритмов}

Алгоритмы тестировались при помощи функции time() из библиотеки time языка Python. Данная функция возвращает количество секунд, прошедших с начала эпохи.

Контрольная точка возвращаемого значения не определна, поэтому допустима только разница между результатами последовательных вызовов.

Замеры времени для каждой длины входного массива проводились 1000 раз. В качестве результата взято среднее время работы алгоритма на данной длине слова. При каждом запуске алгоритма, на вход подавались случайно сгенерированные строки. Тестовые пакеты создавались до начала замера времени.

Результаты замеров приведены в таблицах \ref{tbl:best}, \ref{tbl:wor} и \ref{tbl:random}.

\begin{table}[h]
	\begin{center}
		\captionsetup{justification=raggedright,singlelinecheck=off}
		\caption{\label{tbl:best}Результаты замеров времени для отсортированного массива (в микросекундах).}
		\begin{tabular}{|l|l|l|l|}
			
			\hline
			Длина массива&Пирамидальная&Плавная&Бинарным деревом\\
			\hline
			1 & 1 & 2 & 1 \\
			\hline
			26 & 60 & 55 & 86 \\
			\hline
			51 & 85 & 73 & 199 \\
			\hline
			76 & 129 & 105 & 406 \\
			\hline
			101 & 186 & 139 & 703 \\
			\hline
			126 & 236 & 172 & 1076 \\
			\hline
			151 & 296 & 208 & 1536 \\
			\hline
			176 & 358 & 245 & 2085 \\
			\hline
			201 & 422 & 276 & 2699 \\
			\hline
			
		\end{tabular}
	\end{center}
\end{table}

\begin{table}[h]
	\begin{center}
		\captionsetup{justification=raggedright,singlelinecheck=off}
		\caption{\label{tbl:wor}Результаты замеров времени для отсортированного в обратном порядке массива (в микросекундах).}
		\begin{tabular}{|l|l|l|l|}
			
			\hline
			Длина массива&Пирамидальная&Плавная&Бинарным деревом\\
			\hline
			1 & 1 & 2 & 1 \\
			\hline
			26 & 48 & 141 & 90 \\
			\hline
			51 & 68 & 208 & 188 \\
			\hline
			76 & 109 & 315 & 403 \\
			\hline
			101 & 156 & 437 & 686 \\
			\hline
			126 & 202 & 578 & 1059 \\
			\hline
			151 & 254 & 730 & 1547 \\
			\hline
			176 & 307 & 849 & 2058 \\
			\hline
			201 & 357 & 992 & 2868 \\
			\hline
			
		\end{tabular}
	\end{center}
\end{table}

\begin{table}[h]
	\begin{center}
		\captionsetup{justification=raggedright,singlelinecheck=off}
		\caption{\label{tbl:random}Результаты замеров времени для случайных данных (в микросекундах).}
		\begin{tabular}{|l|l|l|l|}
			
			\hline
			Длина массива&Пирамидальная&Плавная&Бинарным деревом\\
			\hline
			1 & 1 & 2 & 2 \\
			\hline
			26 & 60 & 109 & 66 \\
			\hline
			51 & 90 & 202 & 77 \\
			\hline
			76 & 145 & 324 & 118 \\
			\hline
			101 & 203 & 454 & 164 \\
			\hline
			126 & 260 & 593 & 210 \\
			\hline
			151 & 324 & 736 & 260 \\
			\hline
			176 & 388 & 886 & 309 \\
			\hline
			201 & 456 & 1036 & 366 \\
			\hline
			
		\end{tabular}
	\end{center}
\end{table}
\clearpage

\section*{Вывод}

Алгоритм плавной сортировки работает быстрее на изначально упорядоченном массиве, чем остальные алгоритмы. Алгоритм пирамидальной сортировки работает быстрее в случае, если данные упорядоченны в обратном направлении. Алгоритм сортировки бинарным деревом работает быстрее на случайно упорядоченных данных.


