\chapter{Технологическая часть}

В данном разделе будут приведены требования к программному обеспечению, средства реализации и листинги кода.

\section{Требования к программе}

К программе предъявляется ряд требований:
\begin{itemize}
	\item на вход подаётся массив сравнимых элементов (целые числа);
	\item на выходе — тот же массив, но в отсортированном порядке;
	\item программа не должна аварийно завершаться при некорректном вводе.
\end{itemize}

\section{Средства реализации}

В качестве языка программирования для реализации данной лабораторной работы был выбран язык программирования (ЯП) Python \cite{pythonlang}. 

Данный ЯП предоставляет широкий набор библиотек и руководств по их использованию, что значительно облегчает выполнения большинства задач в области программирования. 

Время работы алгоритмов было замерено с помощью функции time() из библиотеки time \cite{pythonlangtime}.

\section{Сведения о модулях программы}
Программа состоит из шести модулей:
\begin{enumerate}
	\item beenary\_tree\_sort.py -- файл, содержащий реализацию бинарной сортировки;
	\item heap\_sort.py -- файл, содержащий реализаци пирамидальной сортировки;
	\item smooth\_sort.py -- фаайл, содержащий реализацию плавной сортировки;
	\item perfomance\_tests.py -- файл, содержащий функциональные тесты;
	\item time\_tests.py -- файл, содержащий программу, выполняющую замеры времени работы сортировок;
	\item memory\_tests.py -- файл, содержащий программу, выполняющую замеры выделяемой памяти.
\end{enumerate}


\section{Реализация алгоритмов}
В листингах \ref{lst:heap}, \ref{lst:smooth}, \ref{lst:beenary} представлены реализации алгоритмов сортировок (пирамидальной, плавной и бинарным деревом).

\begin{lstlisting}[label=lst:heap,caption=Алгоритм пирамидальной сортировки]
def heap_sort(data):
	for start in range((len(data) - 2) // 2, -1, -1):
		HeapSift(data, start, len(data) - 1) 
	for end in range(len(data) - 1, 0, -1): 
		data[end], data[0] = data[0], data[end]
		HeapSift(data, 0, end - 1)
	return data

def HeapSift(data, start, end):
	root = start 
	while True:
		child = root * 2 + 1
		if child > end: break 
		if child + 1 <= end and data[child] < data[child + 1]:
			child += 1
		if data[root] < data[child]:
			data[root], data[child] = data[child], data[root]
			root = child
		else:
			break
\end{lstlisting}

\begin{lstlisting}[label=lst:smooth,caption= Алгоритм плавной сортировки]
def smoothsort(lst):
	leo_nums = leonardo_numbers(len(lst))
	heap = []

	for i in range(len(lst)):
		if len(heap) >= 2 and heap[-2] == heap[-1] + 1:
			heap.pop()
			heap[-1] += 1
		else:
			if len(heap) >= 1 and heap[-1] == 1:
				heap.append(0)
			else:
				heap.append(1)
				restore_heap(lst, i, heap, leo_nums)

	for i in reversed(range(len(lst))):
		if heap[-1] < 2:
			heap.pop()
		else:
			k = heap.pop()
			t_r, k_r, t_l, k_l = get_child_trees(i, k, leo_nums)
			heap.append(k_l)
		restore_heap(lst, t_l, heap, leo_nums)
		heap.append(k_r)
		restore_heap(lst, t_r, heap, leo_nums)

def leonardo_numbers(hi):
	a, b = 1, 1
	numbers = []
	while a <= hi:
		numbers.append(a)
		a, b = b, a + b + 1
	return numbers

def restore_heap(lst, i, heap, leo_nums):
	current = len(heap) - 1
	k = heap[current]
	
	while current > 0:
		j = i - leo_nums[k]
		if (lst[j] > lst[i] and
			(k < 2 or lst[j] > lst[i-1] and lst[j] > lst[i-2])):
			lst[i], lst[j] = lst[j], lst[i]
			i = j
			current -= 1
			k = heap[current]
		else:
			break

	while k >= 2:
		t_r, k_r, t_l, k_l = get_child_trees(i, k, leo_nums)
		if lst[i] < lst[t_r] or lst[i] < lst[t_l]:
			if lst[t_r] > lst[t_l]:
				lst[i], lst[t_r] = lst[t_r], lst[i]
				i, k = t_r, k_r
			else:
				lst[i], lst[t_l] = lst[t_l], lst[i]
				i, k = t_l, k_l
		else:
			break

def get_child_trees(i, k, leo_nums):
	t_r, k_r = i - 1, k - 2
	t_l, k_l = t_r - leo_nums[k_r], k - 1
	return t_r, k_r, t_l, k_l

def smooth_sort(data):
	smoothsort(data)
	return data
\end{lstlisting}

\begin{lstlisting}[label=lst:beenary,caption=Алгоритм сортировки бинарным деревом]
class Node:
	def __init__(self, val, left = None, right = None):
		self.val = val
		self.left = left
		self.right = right
	
	def insert(self, val):
		if self.val > val:
			if self.left is None:
				self.left = Node(val)
			else:
				self.left.insert(val)
		else:
			if self.right is None:
				self.right = Node(val)
			else:
				self.right.insert(val)
	
	def get_sorted(self, dst):
		if self.left is not None:
			self.left.get_sorted(dst)
		dst.append(self.val)
		if self.right is not None:
			self.right.get_sorted(dst)
	
	def sort(self):
		dst = []
		self.get_sorted(dst)
		return dst


def beenary_tree_sort(arr):
	if len(arr) == 0:
	return arr

	head = Node(arr[0])
	for i in range(1, len(arr)):
	head.insert(arr[i])

	return head.sort()
\end{lstlisting}

\section{Функциональные тесты}

В таблице \ref{tbl:functional_test} приведены тесты для функций, реализующих алгоритмы сортировки. Тесты пройдены успешно.


\begin{table}[h]
	\begin{center}
		\captionsetup{justification=raggedright,singlelinecheck=off}
		\caption{\label{tbl:functional_test} Функциональные тесты}
		\begin{tabular}{|c|c|c|}
			\hline
			Входной массив & Ожидаемый результат & Результат \\ 
			\hline
			$[1,2,3,4]$ & $[1,2,3,4]$  & $[1,2,3,4]$\\
			$[5,4,3,2,1]$  & $[1,2,3,4,5]$ & $[1,2,3,4,5]$\\
			$[3,2,-5,0,1]$  & $[-5,0,0,2,3]$  & $[-5,0,0,2,3]$\\
			$[4]$  & $[4]$  & $[4]$\\
			$[]$  & $[]$  & $[]$\\
			\hline
		\end{tabular}
	\end{center}
\end{table}


\section*{Вывод}

Были выбраны средства реализации и замера времени выполнения алгоритмов сортировки. Были реализованы алгоритмы трех сортировок и программа, реализующая замер времени работы этих алгоритмов.
