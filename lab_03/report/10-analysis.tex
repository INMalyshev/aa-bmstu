\chapter{Аналитическая часть}
В этом разделе будут представлены описания алгоритмов пирамидальной сортировки, плавной сортировки и сортировки двоичным деревом поиска.

\section{Пирамидальная сортировка}

\textbf{Пирамидальная сортировка \cite{Knut}} -- это метод сортировки сравнением, основанный на такой структуре данных как "двоичная куча". Двоичная куча (binary heap) -- структура данных, позволяющая быстро (за логарифмическое время) добавлять элементы и извлекать элемент с максимальным приоритетом (например, максимальный по значению).


Общие идеи алгоритма:
\begin{enumerate}[label=\arabic*)]
	\item построить двоичную кучу из входных данных;
	\item поменять местами первый и последний элемент кучи;
	\item уменьшить размер кучи на 1;
	\item если размер кучи больше 0, перейти к пункту 1.
\end{enumerate}


\section{Плавная сортировка}

\textbf{Плавная сортировка \cite{Knut}} -- алгоритм сортировки выбором, разновидность пирамидальной сортировки. От класиического алгоритма пирамидальной сортировки отличается тем, что его сложность зависит от степени изначальной упорядоченности входного массива, на основе которого строится "двоичная куча".


Шаги алгоритма:
\begin{enumerate}[label=\arabic*)]
	\item сформировать последовательность куч;
	\item сформировать отсортированный массив.
\end{enumerate}

\section{Сортировка бинарным деревом}

\textbf{Сортировка бинарным деревом \cite{Knut}} -- универсальный алгоритм сортировки, заключающийся в построении двоичного дерева поиска по ключам массива, с последующей сборкой результирующего массива путём обхода узлов построенного дерева в необходимом порядке следования ключей.


Шаги алгоритма:
\begin{enumerate}[label=\arabic*)]
	\item построить двоичное дерево поиска по ключам массива;
	\item собрать результирующий массив путём обхода узлов дерева поиска в необходимом порядке следования ключей;
	\item вернуть, в качестве результата, отсортированный массив.
\end{enumerate}

\section*{Вывод}

Необходимо реализовать три алгоритма сортировки: пирамидальную сортировку, плавную сортировку и сортировку бинарным деревом, а также выполнить теоретическую оценку сложности этих алгоритмов и сравнить ее с эксперементальными показателями.



